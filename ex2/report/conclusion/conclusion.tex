\section{Conclusion}

The LEDS on the gamepad lights up as intended with both the usage of polling and interrupts when pushed. Multiple ways of energy conservatism was used. And the program managed to solve this task in a very energy efficient way.

The tests were done with energy mode 1 and energy mode 2. There exists an energy mode 3 which conserves 0.2 \si{\micro\ampere} less than energy mode 3, see reference manual \cite{efm32gg-ref-man} on page 8. We were not sure if we managed to get into this state, and since the power difference is less than the flucutation when the device is idle, it was very hard to know for sure. The biggest gain in power efficiency is to go from energy mode 1 to energy mode 2, and we did this.

We have, by completing this assignment, gained a greater understanding of not only how to influence the energy efficience, but also of assembly programming, the EFM32 microcontroller, and the GNU toolchain in general. Furthermore, it became clear from the test results that using interrupts and letting the CPU sleep deep when there was no I/O to be processed was a improvement, in regards to power-consumption, over polling.
