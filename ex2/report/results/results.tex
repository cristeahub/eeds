\section{Results and Tests}

\subsection{Sounds}

Three different main sounds were implemented, playing on different button pushes. A button never played the same sound, however, as they were semi-random, given some pre-defined constraints. Our first sound was a coin sound, and given spesific ranges for amplitude, frequency (...), we were able to produce different coin sounds each time the same button is pressed.

\subsection{Comparison}

The comparison of the different values are shown in table \ref{tab:energy_usage}.
One major point one can see from the table is that the biggest difference is between energy mode 1 and energy mode 2. This difference is way lager than energy mode 1 and polling. An interrupt in energy mode 2 is 1617 times less power consuming than polling or interrupts in energy mode 1.
A more detailed view of each energy level is shown in appendix B on page \pageref{appendix:b}.

\begin{table}[ht!]
    \begin{center}
    \begin{tabular}{ | l | l | l | }
        \hline
        I/O Mode    & Idling energy usage \\
        \hline
        Polling & 3.51 mA \\
        \hline
        Interrupts w/Energy mode 1 & 1.19 mA \\
        \hline
        Interrupts w/Energy mode 2 & 2.17 \si{\micro\ampere} \\
        \hline
    \end{tabular}
    \caption{Average current while idling for different methods of input reading}
    \label{tab:energy_usage}
    \end{center}
\end{table}

