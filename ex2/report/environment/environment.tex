\section{Development environment}
We used all of the tools from the previous assignment \cite{report-1}, namely \texttt{vim}, \texttt{git}, GitHub, Make and the development stack for EFM32GG.

The same compendium \cite{eeds-compendium} from last exercise was used.
Accompanying this paper was mostly the EFM32GG Reference Manual \cite{efm32gg-ref-man} and the Energy Optimization Application Note \cite{efm32gg-energy-op}.
Since playing sound is a complex task the manuals became more important as we wanted to optimize our solution as much as possible.

In addition to this, some new tools was used which will be outlined here.

\subsection{C Programming language}
The C programming language was used in this assingment.
This programming language serves as an abstraction above Assembly while still being close to the hardware.
By using C for our development the development time was reduced significantly and debugging became much easier.
Like the previous assignment, our compiled code ran right on the hardware.
Because of this we had to write the interrput handeling in C.

\subsection{GCC}
The GNU compiler collection is a compiler used to compile the C code down to bytecode.
It also serves as a linker for the C files. This means the startup code and other useful libraries can easily be incorporated into this project.

\subsection{KOSS UR5}
For listening to the sounds generated by the DAC, we used KOSS UR5 headphones, provided by the lab.
The headphones were connected to the EFM32 microcontroller, and we could through these listen to the analog audio generated by the DAC.
The KOSS UR5 headphones have a frequency range from 80Hz - 18KHz, which is good enough for the purpose of this assignment, even though human hearing range is averaging from 20Hz to 20KHz.
