\section{Development process and implementation}
\label{chap:development_process}

This chapter details the steps used during the development of the program.
We started by re-implementing most of the code used in the previous exercise, only this time in C.
In addition we set up a timer, which was to be used for interacting with the DAC.

% - setting up interrupts from c
% - embedding assembly

\subsection{Setting up the DAC}

To get sound playing on the board, we had to use the DAC, a Digital-to-Analog Converter.
A timer was needed to be able to continuously write sound samples to the DAC.

\subsection{Timer}

The timer clock frequency is 14MHz, but for the music we only want a frequency of 44100Hz.
Therefore we set the sample period to $ \frac{\SI{14}{\mega\hertz}}{\SI{44100}{\hertz}} \approx 317 $.
This will ensure the timer interrupt is triggered 44100 times a second.

\subsection{Sound synthesis}
The sound effects we've implemented are all generated in real-time on the board.
We made four different sound effects: a coin blip, a lasera level-up sound, and an alert.
They are all based on a simple waveform, either a square wave or a sawtooth.
Each sound has parameters that are allowed to change for each playback.
Those parameters include frequency, slide, and ADSR parameters.
To leverage the fact that the sound samples are generated on-the-fly, we added a bit of randomization to the ADSR parameters.
This made each playback of a sound effect unique, so each time you press the button for a given sound effect it will sound slightly different.
In addition to varying the ADSR parameters, the frequency was randomized within a given interval for each of the sound effects.
We also added a custom slide to each of them – this allowed the frequency to change gradually during the sound.

\subsubsection{ADSR}

A waveform with constant frequency alone would not provide the sound effects we wanted. We therefore implemented a ADSR envelope \cite{adsr}.
The envelope consists of an attack period, a decay period, a sustain level and a release period, as seen in figure \ref{fig:adsr_envelope}.

\begin{itemize}
    \item The attack periode is the time to use from zero to maximum amplitude.
    \item The decay is the time from max amplitude down to the sustain level.
    \item The sustain level is the amplitude to rest on between the decay and release periods.
    \item The release indicates how much time to spend fading from the sustain level to zero volume.
\end{itemize}

\begin{figure}[ht!]
    \begin{center}
    \includegraphics[width=0.8\textwidth]{assets/img/adsr.png}
    \caption{A ADSR envelope}
    \label{fig:adsr_envelope}
    \end{center}
\end{figure}

\subsubsection{Effects}

The parameter intervals for the four implemented sound effects can be seen in Table \ref{tab:sound_effects}.

\begin{table}[ht!]
    \begin{center}
    \begin{tabular}{r|llll}
    Parameter         & Coin       & Laser             & Level-up   & Alert      \\
    \hline
    Waveform          & Square     & Square / Sawtooth & Sawtooth   & Sawtooth   \\
    Initial frequency & 1500-3000  & 1200-3200         & 100-1100   & 175-255    \\
    Slide             & 0          & -2                & 3-6        & 0          \\
    Attack            & 0          & 0                 & 0-500      & 0          \\
    Decay             & 0-3500     & 1000-3000         & 1000-3000  & 500-1500   \\
    Sustain level     & 30-80      & 0-50              & 50-100     & 50-100     \\
    Release           & 4000-18000 & 4500-10500        & 2500-16500 & 2500-16500 \\
    \end{tabular}
    \end{center}
    \caption{Parameter intervals for the implemented sound effects}
    \label{tab:sound_effects}
\end{table}

\begin{figure}[ht!]
\begin{code}
void generate_coin() {
    current_sound = (Sound){
        .freq = rand() % 1500 + 1500,
        .a = 0,
        .d = rand() % 3500,
        .s = rand() % 50 + 30,
        .r = rant() % 14000 + 4000,
        .slide = 0,
        .wave = int_square
    };
    i = 0;
}
\end{code}
\caption{Code for generating a coin pick sound}
\label{fig:coin}
\end{figure}

All sound effect calculations are done with integers, to ensure that the calculations are fast enough to be processed within one timer interval.
For each effect, there's a function to initalize a struct with the sounds custom parameters. This funtion will be called each time the sound's corresponding button is pressed. An example of this may be seen in Figure \ref{fig:coin}.

\subsubsection{Custom random function}

There was no rand() function available, so we made one ourselves.

\begin{figure}[!ht]
\begin{code}

static unsigned int next = 1;

int rand_r(unsigned int *seed) {
    *seed = *seed * 1103515245 + 12345;
    return (*seed % ((unsigned int)RAND_MAX + 1));
}

int rand() {
    return (rand_r(&next));
}

\end{code}
\caption{The random function}
\end{figure}


\subsection{Pre-recorded melody}

For the start-up melody, we chose to produce a song using FL Studio, a digital audio workstation.
We would then export the .wav file and convert it to a C array, as a wav file consists of two channels of typically 44,100 samples per second, and 8 bits per sample.
We had to first export the song to a 16-bit int .wav file, as FL Studio didn't accept 8-bit uint .wavformat.
We opened the song in Audacity, as this audio editor allowed us to export it to a 8 bit int .wav file.
Moreover, we used a Python library to convert the .wav samples to a C array, tailoring this general Python .wav interpreter code \cite{wav} to fit our needs. Our edited Python code to produce the C file that holds the two arrays that represents the two channels is shown in figure \ref{fig:python} in appendix A.

The sound clip we made originally lasted around 15 seconds, which results in $\SI{44100}{Hz} * 15 \text{seconds} = 661500 \text{samples}$.
Since this file was too large for the DAC, we shortened the song to 8 seconds.
This wav-file amounted to 704 kB of space, which became 3.5 MB when converted to two arrays in C. This proved to be small enough for the DAC to interpret.

The start up melody can be found here. \cite{song}

\subsection{Button control}

Buttons are set up to control one sound effect each. The remaining buttons will stop the sound currently playing.
When a GPIO interrupt is triggered, we check whether any buttons are being pressed down.
The first button which is currently being pressed will have its designated action executed.
When one sound is initiated, any currently playing sound is cancelled immediately.

\subsection{Energy consumption}

The desire for this task was to be able to use as little power as possible.
To achieve this we put a great deal of work into ensuring the program was put in the most energy saving state possible for the current task.
This means that even though music was played, the card was still in an energy saving mode.

For maximum efficiency the board dives straight into deep sleep after startup.
One could agrue that a small opening sequence would be useful for the end user, but for this task the startup time could instead be instant.
Since there are no definitive answer to this we thought we should go for maximum efficiency this time.

Every interaction with the board gives a feedback.
We made sure that the board goes back to sleep as soon as possible after the given task is done.
When a song is played the interrupt handler wakes up the board and starts to play the song.
If a new song is being played while a current song is playing, the current task is thrown away, and the new song starts.
This technique ensures that no power is wasted on music the end user doesn't want.

When the song is over the card by itself goes back into deep sleep and waits for another interrupt.

The one thing that does use a lot of power is computing generated sounds.
If one wanted to make a more energy efficient card, one could include more songs for playback, and not generate them live.
