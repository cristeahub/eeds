\section{Development process and implementation}
\label{chap:development_process}

This chapter details the steps used during the development of the program. Features of the program were added iteratively, beginning with simply getting the LEDs to light up and then evolving to more sophisticated funcionality. The first iteration of the code simply turned on the LEDs. After this, an opening sequence was added to the program, the details of which and why it was added is explained more in section \ref{subsec:dev_pros_opening_seq}. Afterwards, a polling technique was used to test the buttons on the game pad. With the correct flags set a number will be written to a specific location in memory which indicates what buttons are being pressed. This number is then shifted eight places to the left to comply with the way LEDs are lightened up.

The next step was to make interrupts work. A main function was constructed to set the correct flags. When these are set the device goes to sleep and waits for an interrupt. A main focus was to abort the interrupts as quickly as possible. More on this in section \ref{subsec:dev_pros_interrupts}.

\subsection{Pre-assignment setup}
\label{subsec:pre-assingment_setup}

The assignment involves a lot of GPIO usage, and therefore some values are used over and over. The register design is set up so that three registers are storing GPIO constants and one register is used for simulating a wait for the opening sequence. They each have an alias to be easier to access. GPIO\_PA\_BASE was renamed gpio\_o and GPIO\_PC\_BASE was renamed to gpio\_i, GPIO\_BASE was renamed gpio. The remaining registers from r0 to r7 is used for variable content. See table \ref{tab:register_design} in appendix A on page \pageref{tab:register_design} for an overview of the registers.

\subsection{Setting up the DAC}

To get sound playing on the board, we had to use the DAC, a Digital-to-Analog Converter.
A timer was needed to be able to continuosly write sound samples to the DAC.

\subsection{Sound synthesis}
The sound effects we've implemented are all generated in real-time on the board.
We made four different sound effects: a coin blip, a lasera level-up sound, and an alert.
They are all based on a simple waveform, either a square wave or a sawtooth.
Each sound has parameters that are allowed to change for each playback.
Those parameters include frequency, slide, and ADSR parameters.
To leverage the fact that the sound samples are generated on-the-fly, we added a bit of randomization to the ADSR parameters.
This made each playback of a sound effect unique, so each time you press the button for a given sound effect it will sound slightly different.
In addition to varying the ADSR parameters, the frequency was randomized within a given interval for each of the sound effects.
We also added a custom slide to each of them – this allowed the frequency to change gradually during the sound.

\subsubsection{ADSR}

A waveform with constant frequency alone would not provide the sound effects we wanted. We therefore implemented a ADSR envelope \cite{adsr}.
The envelope consists of an attack period, a decay period, a sustain level and a release period, as seen in figure \ref{fig:adsr_envelope}.

\begin{itemize}
    \item The attack periode is the time to use from zero to maximum amplitude.
    \item The decay is the time from max amplitude down to the sustain level.
    \item The sustain level is the amplitude to rest on between the decay and release periods.
    \item The release indicates how much time to spend fading from the sustain level to zero volume.
\end{itemize}

\begin{figure}[ht!]
    \begin{center}
    \includegraphics[width=0.8\textwidth]{assets/img/adsr.png}
    \caption{A ADSR envelope}
    \label{fig:adsr_envelope}
    \end{center}
\end{figure}

\subsubsection{Effects}

The parameter intervals for the four implemented sound effects can be seen in Table \ref{tab:sound_effects}.

\begin{table}[ht!]
    \begin{center}
    \begin{tabular}{r|llll}
    Parameter         & Coin       & Laser             & Level-up   & Alert      \\
    \hline
    Waveform          & Square     & Square / Sawtooth & Sawtooth   & Sawtooth   \\
    Initial frequency & 1500-3000  & 1200-3200         & 100-1100   & 175-255    \\
    Slide             & 0          & -2                & 3-6        & 0          \\
    Attack            & 0          & 0                 & 0-500      & 0          \\
    Decay             & 0-3500     & 1000-3000         & 1000-3000  & 500-1500   \\
    Sustain level     & 30-80      & 0-50              & 50-100     & 50-100     \\
    Release           & 4000-18000 & 4500-10500        & 2500-16500 & 2500-16500 \\
    \end{tabular}
    \end{center}
    \caption{Parameter intervals for the implemented sound effects}
    \label{tab:sound_effects}
\end{table}

\begin{figure}[ht!]
\begin{code}
void generate_coin() {
    current_sound = (Sound){
        .freq = rand() % 1500 + 1500,
        .a = 0,
        .d = rand() % 3500,
        .s = rand() % 50 + 30,
        .r = rant() % 14000 + 4000,
        .slide = 0,
        .wave = int_square
    };
    i = 0;
}
\end{code}
\caption{Code for generating a coin pick sound}
\label{fig:coin}
\end{figure}

All sound effect calculations are done with integers, to ensure that the calculations are fast enough to be processed within one timer interval.
For each effect, there's a function to initalize a struct with the sounds custom parameters. This funtion will be called each time the sound's corresponding button is pressed. An example of this may be seen in Figure \ref{fig:coin}.

\subsubsection{Custom random function}

There was no rand() function available, so we made one ourselves.

\begin{figure}[!ht]
\begin{code}

static unsigned int next = 1;

int rand_r(unsigned int *seed) {
    *seed = *seed * 1103515245 + 12345;
    return (*seed % ((unsigned int)RAND_MAX + 1));
}

int rand() {
    return (rand_r(&next));
}

\end{code}
\caption{The random function}
\end{figure}


\subsection{Pre-recorded melody}

% Details of exporting from FL Studio, convert to C array

\subsection{Button control}

Buttons are set up to control one sound effect each. The remaining buttons will stop the sound currently playing.
