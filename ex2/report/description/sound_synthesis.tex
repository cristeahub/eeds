\subsection{Sound synthesis}
The sound effects we've implemented are all generated in real-time on the board.
We made four different sound effects: a coin blip, a lasera level-up sound, and an alert.
They are all based on a simple waveform, either a square wave or a sawtooth.
Each sound has parameters that are allowed to change for each playback.
Those parameters include frequency, slide, and ADSR parameters.
To leverage the fact that the sound samples are generated on-the-fly, we added a bit of randomization to the ADSR parameters.
This made each playback of a sound effect unique, so each time you press the button for a given sound effect it will sound slightly different.
In addition to varying the ADSR parameters, the frequency was randomized within a given interval for each of the sound effects.
We also added a custom slide to each of them – this allowed the frequency to change gradually during the sound.

\subsubsection{ADSR}

A waveform with constant frequency alone would not provide the sound effects we wanted. We therefore implemented a ADSR envelope \cite{adsr}.
The envelope consists of an attack period, a decay period, a sustain level and a release period, as seen in figure \ref{fig:adsr_envelope}.

\begin{itemize}
    \item The attack periode is the time to use from zero to maximum amplitude.
    \item The decay is the time from max amplitude down to the sustain level.
    \item The sustain level is the amplitude to rest on between the decay and release periods.
    \item The release indicates how much time to spend fading from the sustain level to zero volume.
\end{itemize}

\begin{figure}[ht!]
    \begin{center}
    \includegraphics[width=0.8\textwidth]{assets/img/adsr.png}
    \caption{A ADSR envelope}
    \label{fig:adsr_envelope}
    \end{center}
\end{figure}

\subsubsection{Effects}

The parameter intervals for the four implemented sound effects can be seen in Table \ref{tab:sound_effects}.

\begin{table}[ht!]
    \begin{center}
    \begin{tabular}{r|llll}
    Parameter         & Coin       & Laser             & Level-up   & Alert      \\
    \hline
    Waveform          & Square     & Square / Sawtooth & Sawtooth   & Sawtooth   \\
    Initial frequency & 1500-3000  & 1200-3200         & 100-1100   & 175-255    \\
    Slide             & 0          & -2                & 3-6        & 0          \\
    Attack            & 0          & 0                 & 0-500      & 0          \\
    Decay             & 0-3500     & 1000-3000         & 1000-3000  & 500-1500   \\
    Sustain level     & 30-80      & 0-50              & 50-100     & 50-100     \\
    Release           & 4000-18000 & 4500-10500        & 2500-16500 & 2500-16500 \\
    \end{tabular}
    \end{center}
    \caption{Parameter intervals for the implemented sound effects}
    \label{tab:sound_effects}
\end{table}

\begin{figure}[ht!]
\begin{code}
void generate_coin() {
    current_sound = (Sound){
        .freq = rand() % 1500 + 1500,
        .a = 0,
        .d = rand() % 3500,
        .s = rand() % 50 + 30,
        .r = rant() % 14000 + 4000,
        .slide = 0,
        .wave = int_square
    };
    i = 0;
}
\end{code}
\caption{Code for generating a coin pick sound}
\label{fig:coin}
\end{figure}

All sound effect calculations are done with integers, to ensure that the calculations are fast enough to be processed within one timer interval.
For each effect, there's a function to initalize a struct with the sounds custom parameters. This funtion will be called each time the sound's corresponding button is pressed. An example of this may be seen in Figure \ref{fig:coin}.

\subsubsection{Custom random function}

There was no rand() function available, so we made one ourselves.

\begin{figure}[!ht]
\begin{code}

static unsigned int next = 1;

int rand_r(unsigned int *seed) {
    *seed = *seed * 1103515245 + 12345;
    return (*seed % ((unsigned int)RAND_MAX + 1));
}

int rand() {
    return (rand_r(&next));
}

\end{code}
\caption{The random function}
\end{figure}
