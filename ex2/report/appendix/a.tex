\section{Appendix A}
\label{appenix:a}

This appendix includes supplimentary tables for the report.

\begin{figure}[ht!]
\begin{code}
import wave, struct

def pcm_channels(wave_file):
    stream = wave.open(wave_file,"rb")

    num_channels = stream.getnchannels()
    sample_rate = stream.getframerate()
    sample_width = stream.getsampwidth()
    num_frames = stream.getnframes()

    raw_data = stream.readframes( num_frames )
    stream.close()

    total_samples = num_frames * num_channels

    if sample_width == 1: 
        fmt = "%iB" % total_samples
    elif sample_width == 2:
        fmt = "%ih" % total_samples
    else:
        raise ValueError("Only supports 8 and 16 bit audio formats.")

    integer_data = struct.unpack(fmt, raw_data)
    del raw_data

    channels = [ [] for time in range(num_channels) ]

    for index, value in enumerate(integer_data):
        bucket = index % num_channels
        channels[bucket].append(value)

    return channels

f = open('song.c', 'w')

chan0, chan1 = pcm_channels('file.wav')[0], pcm_channels('file.wav')[1]

#Outputs the C code with the two arrays
f.write('#include <stdint.h>\n\nconst uint8_t channel0[] = {' + str(chan0)[1:-1]\
 + '};\n\nconst uint8_t channel1[] = {' + str(chan1)[1:-1] + '};')

\end{code}
\caption{Python code for converting a wav file to an array of integers}
\label{fig:python}
\end{figure}
