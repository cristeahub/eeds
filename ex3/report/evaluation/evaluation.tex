\section{Evaluation of Assignment}

The ability to combine the skills we have acquired throughout the course into one project was a good experience.
Making a game for the EFM32 turned out to be a good way to get an introduction to linux device drivers.
It also greatly improved our skills with the C programming language.
The compendium \cite{eeds-compendium} provided us with useful information, and the provided skeleton code was a useful resource while writing the gamepad driver and the game, which was great.

The last three days before the deadline, the computers were really slow.
Moreso than normal when working on a network drive.
On some occasions, it literally took hours to finish a task of writing to multiple files.
This made it really frustrating to work on the lab, as there was a lot of time where we just had to sit and wait.
Information was given to move our working files to /tmp/ to work with.
This is more of a work-around than a solution.

The conditions in lab 458 are extremely bad for humans.
Not only is the ventilation non-existing, but the number of people taking the course this year made this even worse.
Students regularly have to take breathing breaks not to faint and a headache is guaranteed.
This is a huge problem for both students and teachers, and should be dealt with as soon as possible.

Overall this subject is one of the more interesting on NTNU.
The uniqueness of the tasks supplied in this subject is highly appreciated.
There are too few hardware oriented subjects that give such an insight into embedded systems.
