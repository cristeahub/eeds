\section{Results and Tests}

\subsection{Energy consumption}

As long as no music is being played, the energy consumption is low. This is because the board goes into deep sleep when idling, see figure \ref{fig:idle} on page \pageref{fig:idle}.
It spikes upon button presses, when music is being played, see figure \ref{fig:song_and_songs} and figure \ref{fig:sounds} on page \pageref{fig:song_and_songs}.

There are two types of sound playback.
Either the prerendered wav-file \ref{fig:playback_song}, or the on-the-fly generated sound effects \ref{fig:sounds}.

The difference in energy consumption between these two tasks is huge.
When the sounds are generated, multiple mathematical operations needs to be performed, which uses a lot of power.
In difference to this, playback is just a lookup in a pregenerated table, and thus uses a lot less power.

The differences can easier be seen in the table below.

\begin{table}[ht!]
    \begin{center}
    \begin{tabular}{ | l | l | }
        \hline
        Instance            & Average energy usage \\
        \hline
        Idle                & 1.93 \si{\micro\ampere} \\
        \hline
        Song playback       & 2.50 \si{\milli\ampere} \\
        \hline
        Sound generation    & 5.25 \si{\milli\ampere} \\
        \hline
    \end{tabular}
    \caption{Difference in energy usage between states}
    \label{tab:energy_usage}
    \end{center}
\end{table}
