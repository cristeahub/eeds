\section{Development environment}
We used all of the tools from the two previous assignments \cite{report-1}, namely \texttt{vim}, \texttt{git}, GitHub, Make and the development stack for EFM32GG.

Moreover, we also used the same compendium \cite{eeds-compendium}, more precisely the information given in section 5 (Exercise 3).
Accompanying the compendium was mostly the EFM32GG Reference Manual \cite{efm32gg-ref-man}, the Energy Optimization Application Note \cite{efm32gg-energy-op}, also used in the previous assignments. In this assignment, we mainly used the documentation for the ptxdist build system and the \si{\micro}Clinux documentation.

The additional resources (that was not mentioned in the previous assignments) are outlined below. We were also provided with a useful skeleton code.

\subsection{\si{\micro}Clinux}
The \si{\micro}Clinux is a variant of the Linux operating system for microcontrollers (hence the name \si{\micro}C: microController). Moreover, it was originally a derivative, or a fork, of the Linux 2.0 kernel.
It is intented for microcontrollers that does not have a Memory Management Unit.

\subsection{ptxdist}
The Linux distribution is built using the build system ptxdist, which will package the kernel and other software (after compiling it) into binary files that are flashed on the development board.
The ptxdist documentation as well as the compendium (in particular Table 5.1) provided us with the information needed.

