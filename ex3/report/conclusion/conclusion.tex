\section{Conclusion}

In doing this assignment, we learned how to use the ptxdist build system with the \si{\micro}Clinux fork of the linux kernel.
We got to use the knowledge we obtained from the previous two assignments to make a game and write a driver for the gamepad.
We chose to make a Tetris based puzzle, as we wanted to keep elements from a classic game, but also make our new twist on it, so that it becomes somewhat original.
We found this assignment to be really educational, and it was also fun getting to implement our own game on a microcontroller.

\subsection{Further improvements}

Each time a place is either moved, rotated or placed, we redraw the game board.
To achieve even greater energy efficience than we did, we could make sure to only re-draw the least possible amount of pixels that is changed on the screen.
This could be done by first drawing the area where the piece was located before the action, and then drawing the area where the piece is located after the action.

It's very hard to get any deeper sleep with an operating system running.
One solution to this would be to implement the game without the OS.
Either by making our own screen renderer that printed to the screen, or by using a more lightweight library.

\subsection{High score}

Here follows a high-score list with our scores. Can you beat us?

\begin{table}[ht!]
        \begin{tabular}{ll}
                Christoffer & 380 \\
                Stian       & 378 \\
                Asbjørn     & 362 \\
                \end{tabular}
            \end{table}

