\section{Results and Tests}

There was an opening sequence and two versions of the program implemented; one version using polling, where the computer has to check the status of the GPIO repeatedly and the final version using interrputs, where the program may sleep between its tasks. The version running interrupts is expected to produce significantly less power consumptive results. This section addresses the implications of an opening sequence as well as results for both programs.

\subsection{Implications of an opening sequence}

When no opening sequence is present, the power consumption of the device is approximately 370 \si{\micro\ampere} over very few clock cycles, see Figure \ref{fig:no_opening_seq} in appendix B on site \pageref{fig:no_opening_seq}. When there is an opening sequence, the power consumption for the device is approximately 4.3 \si{\milli\ampere} over multiple seconds, see Figure \ref{fig:opening_seq} in appendix B on page \pageref{fig:opening_seq}. Because of this, there is a significant power usage if the opening sequence is present.

The opening sequence is meant for the user. Most of the devices with an architecture like the one we are programming for, are devices that needs to live a long time with mostly no interaction, but when they are told to work, it's crucial that they do. Therefore an opening sequence will be beneficial. If one of the LEDs in this case is faulty, one can see it when booting up the device rather than later. It might be, if the device is designed for an alarm, that the alam will never play and no feedback will be given.

The opening sequence uses a lot of power, but this is mostly due to its long duration. One way to improve it would be to shorten it while still testing all LEDs in a fashion easy understandable by a human being. If the system is supposed to be turned on for a long time, for instance multiple years, then the power consumption of the opening sequence in this case is not so crucial. The longer the device is on, the more negligible the opening sequence becomes.

\subsection{Polling}

See Figure \ref{fig:polling_io} in the appendices for the power consumption while using polling. The energy usage is constantly quite high, on about 3.51 mA. The difference in power usage between input processing and when there's no input is minimal.

Average current while idling: 3.51 mA.

\subsection{Interrupt}

Interrupts should result in a more energy-friendly power usage than polling.
Because the CPU can be woken up from sleep upon interrupts, it is able to sleep and use very small amounts of power when there is no processing to be done. Interrupt driven I/O has been implemented with two different energy modes during sleeping.

\subsubsection{With energy mode 1}

See Figure \ref{fig:interrupt_io} in the appendices for the power consumption while using interrupts without deep sleep.

Average current while idling: 1.19 mA.

\subsubsection{With energy mode 2}

See Figure \ref{fig:interrupt_io_deep_sleep} in the appendices for the power consumption while using interrupts with energy mode 2. The different stages of the program are marked as follows:

Average current while idling: 2.17 \si{\micro\ampere}.

\subsection{Comparison}

See Figure \ref{tab:comparison} in Appendix A for the comparison of the values. One can see that the difference between interrupt with energy mode 1 and interrupt with energy mode 2 is way larger than the difference between interrupt with energy mode 1 and polling. According to our test results, interrupt with energy mode 1 is three times less power consuming than polling, and interrupting with energy mode 2 is an impressive 1617 times, or three orders of magnitude, less power consuming than interrupt with energy mode 1!
