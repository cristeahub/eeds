\section{Results and Tests}

As mentioned in the description section, we implemented two versions of this program; the first one using polling, where the computer has to check the status of the GPIO repeatedly, and the final version using interrupts, where the program may sleep between its tasks. The version running interrupts is, of course, expected to produce significantly less power consumptive results. Below follows test results for both programs, with a summarizing comparison at the end.

\subsection{Polling}

See figure \ref{fig:polling_io} in the appendices for the power consumption while using polling. The energy usage is constantly quite high, on about 3.51 mA. The difference in power usage between input processing and when there's no input is minimal.

Average current while idling: 3.51 mA.

\subsection{Interrupt}

Interrupts should result in a more energy-friendly power usage than polling.
Because the CPU can be woken up from sleep upon interrupts, it is able to sleep and use very small amounts of power when there is no processing to be done. Interrupt driven I/O has been implemented with two different energy modes during sleeping.

\subsubsection{With energy mode 1}

See figure \ref{fig:interrupt_io} in the appendices for the power consumption while using interrupts without deep sleep.

Average current while idling: 1.19 mA.

\subsubsection{With energy mode 2}

See figure \ref{fig:interrupt_io_deep_sleep} in the appendices for the power consumption while using interrupts with energy mode 2. The different stages of the program are marked as follows:

Average current while idling: 2.17$\mu$A.

\subsection{Comparison}

See figure \ref{tab:comparison} Appendix A for the comparison of the values.
