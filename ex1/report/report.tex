\documentclass[paper=a4, fontsize=11pt]{scrartcl} % A4 paper and 11pt font size
\usepackage[utf8]{inputenc}
\usepackage[T1]{fontenc} % Use 8-bit encoding that has 256 glyphs
\usepackage[english]{babel} % English language/hyphenation
\usepackage{amsmath,amsfonts,amsthm} % Math packages

\usepackage{graphicx}
\usepackage{siunitx}

\usepackage{lipsum} % Used for inserting dummy 'Lorem ipsum' text into the template

\usepackage{sectsty} % Allows customizing section commands
\allsectionsfont{\centering \normalfont\scshape} % Make all sections centered, the default font and small caps

\usepackage{fancyhdr} % Custom headers and footers
\pagestyle{fancyplain} % Makes all pages in the document conform to the custom headers and footers
\fancyhead[L]{GROUP 6} % No page header - if you want one, create it in the same way as the footers below
\fancyfoot[L]{} % Empty left footer
\fancyfoot[C]{} % Empty center footer
\fancyfoot[R]{\thepage} % Page numbering for right footer
\renewcommand{\headrulewidth}{0pt} % Remove header underlines
\renewcommand{\footrulewidth}{0pt} % Remove footer underlines
\setlength{\headheight}{13.6pt} % Customize the height of the header

\usepackage{hyperref}
\hypersetup{
    colorlinks,
    citecolor=black,
    filecolor=black,
    linkcolor=black,
    urlcolor=black
}

\input{alphasection.tex}

\numberwithin{equation}{section} % Number equations within sections (i.e. 1.1, 1.2, 2.1, 2.2 instead of 1, 2, 3, 4)
\numberwithin{figure}{section} % Number figures within sections (i.e. 1.1, 1.2, 2.1, 2.2 instead of 1, 2, 3, 4)
\numberwithin{table}{section} % Number tables within sections (i.e. 1.1, 1.2, 2.1, 2.2 instead of 1, 2, 3, 4)

%------ Listings --------------
\usepackage{color}
\definecolor{light-gray}{gray}{0.95}
\usepackage{listings}

\lstnewenvironment{code}[1][]%
{\minipage{\linewidth}
\lstset{ %
language={[x86masm]Assembler},  % choose the language of the code
basicstyle=\footnotesize,       % the size of the fonts that are used for the code
numbers=left,                   % where to put the line-numbers
numberstyle=\footnotesize,      % the size of the fonts that are used for the line-numbers
stepnumber=1,                   % the step between two line-numbers. If it is 1 each line will be numbered
resetmargins=true,              % reset line numbers
numbersep=5pt,                  % how far the line-numbers are from the code
backgroundcolor=\color{white},  % choose the background color. You must add \usepackage{color}
showspaces=false,               % show spaces adding particular underscores
showstringspaces=false,         % underline spaces within strings
showtabs=false,                 % show tabs within strings adding particular underscores
frame=single,                   % adds a frame around the code
tabsize=2,                      % sets default tabsize to 2 spaces
captionpos=b,                   % sets the caption-position to bottom
breaklines=true,                % sets automatic line breaking
breakatwhitespace=false,        % sets if automatic breaks should only happen at whitespace
escapeinside={\%*}{*)},         % if you want to add a comment within your code
morekeywords={
    orr, ldr, bne, subs
    },                          % additional keywords to expand the asm language
#1
}}%
{\endminipage}
%----------------------------------------------------------------------------------------
%   TITLE SECTION
%----------------------------------------------------------------------------------------

\newcommand{\horrule}[1]{\rule{\linewidth}{#1}} % Create horizontal rule command with 1 argument of height

\title{ 
\normalfont \normalsize 
\textsc{Norwegian University of Science and Technology} \\ [25pt] % Your university, school and/or department name(s)
\horrule{0.5pt} \\[0.4cm] % Thin top horizontal rule
\huge \textbf{Assignment 1} \\ % The assignment title
TDT4258 \\
\horrule{2pt} \\[0.5cm] % Thick bottom horizontal rule
}

\author{Stian Jensen\\Asbjørn Ottesen Steinskog\\Christoffer Tønnessen}

\date{\normalsize\today}

\begin{document}

\pagenumbering{roman}

\maketitle

\newpage

%ABSTRACT
\section{Abstract}

This report provides a solution to the second lab exercise in TDT4258 Energy Efficient Computer Systems.
The goal of this exercise was to produce sound effects by pushing different buttons.
A DAC (Digital to Analog Converter) was connected to an amplifier on the DK3750 development board.
The DAC generates analog signals based on values written to a memory area.
By using the EFM32GG development board and writing code in the C programming language, we were able to produce different sounds through sound wave synthesis.


\newpage

\tableofcontents

\setcounter{secnumdepth}{3}

\newpage

\pagenumbering{arabic}

%INTRODUCTION
\section{Introduction}

In this exercise, we made a computer game playable on the EFM32 microcontroller.

We made a puzzle game inspired by Tetris and Bejewelled with simplified rules. We will in this report explain in detail how we wrote the driver for the gamepad, how we made the game, and give an explanation on how we ensured that our implementation ran as energy efficiently as possible. We will then give an evaluation of the assignment and a conclusion that includes statistics about the energy consumption. 


\newpage

%DESCRIPTION AND METHODOLOGY
\section{Development process and implementation}
\label{chap:development_process}

This chapter details the steps used during the development of the program.
We started by re-implementing most of the code used in the previous exercise, only this time in C.
In addition we set up a timer, which was to be used for interacting with the DAC.

% - setting up interrupts from c
% - embedding assembly

\subsection{Setting up the DAC}

To get sound playing on the board, we had to use the DAC, a Digital-to-Analog Converter.
A timer was needed to be able to continuously write sound samples to the DAC.

\subsection{Timer}

The timer clock frequency is 14MHz, but for the music we only want a frequency of 44100Hz.
Therefore we set the sample period to $ \frac{\SI{14}{\mega\hertz}}{\SI{44100}{\hertz}} \approx 317 $.
This will ensure the timer interrupt is triggered 44100 times a second.

\subsection{Sound synthesis}
The sound effects we've implemented are all generated in real-time on the board.
We made four different sound effects: a coin blip, a lasera level-up sound, and an alert.
They are all based on a simple waveform, either a square wave or a sawtooth.
Each sound has parameters that are allowed to change for each playback.
Those parameters include frequency, slide, and ADSR parameters.
To leverage the fact that the sound samples are generated on-the-fly, we added a bit of randomization to the ADSR parameters.
This made each playback of a sound effect unique, so each time you press the button for a given sound effect it will sound slightly different.
In addition to varying the ADSR parameters, the frequency was randomized within a given interval for each of the sound effects.
We also added a custom slide to each of them – this allowed the frequency to change gradually during the sound.

\subsubsection{ADSR}

A waveform with constant frequency alone would not provide the sound effects we wanted. We therefore implemented a ADSR envelope \cite{adsr}.
The envelope consists of an attack period, a decay period, a sustain level and a release period, as seen in figure \ref{fig:adsr_envelope}.

\begin{itemize}
    \item The attack periode is the time to use from zero to maximum amplitude.
    \item The decay is the time from max amplitude down to the sustain level.
    \item The sustain level is the amplitude to rest on between the decay and release periods.
    \item The release indicates how much time to spend fading from the sustain level to zero volume.
\end{itemize}

\begin{figure}[ht!]
    \begin{center}
    \includegraphics[width=0.8\textwidth]{assets/img/adsr.png}
    \caption{A ADSR envelope}
    \label{fig:adsr_envelope}
    \end{center}
\end{figure}

\subsubsection{Effects}

The parameter intervals for the four implemented sound effects can be seen in Table \ref{tab:sound_effects}.

\begin{table}[ht!]
    \begin{center}
    \begin{tabular}{r|llll}
    Parameter         & Coin       & Laser             & Level-up   & Alert      \\
    \hline
    Waveform          & Square     & Square / Sawtooth & Sawtooth   & Sawtooth   \\
    Initial frequency & 1500-3000  & 1200-3200         & 100-1100   & 175-255    \\
    Slide             & 0          & -2                & 3-6        & 0          \\
    Attack            & 0          & 0                 & 0-500      & 0          \\
    Decay             & 0-3500     & 1000-3000         & 1000-3000  & 500-1500   \\
    Sustain level     & 30-80      & 0-50              & 50-100     & 50-100     \\
    Release           & 4000-18000 & 4500-10500        & 2500-16500 & 2500-16500 \\
    \end{tabular}
    \end{center}
    \caption{Parameter intervals for the implemented sound effects}
    \label{tab:sound_effects}
\end{table}

\begin{figure}[ht!]
\begin{code}
void generate_coin() {
    current_sound = (Sound){
        .freq = rand() % 1500 + 1500,
        .a = 0,
        .d = rand() % 3500,
        .s = rand() % 50 + 30,
        .r = rant() % 14000 + 4000,
        .slide = 0,
        .wave = int_square
    };
    i = 0;
}
\end{code}
\caption{Code for generating a coin pick sound}
\label{fig:coin}
\end{figure}

All sound effect calculations are done with integers, to ensure that the calculations are fast enough to be processed within one timer interval.
For each effect, there's a function to initalize a struct with the sounds custom parameters. This funtion will be called each time the sound's corresponding button is pressed. An example of this may be seen in Figure \ref{fig:coin}.

\subsubsection{Custom random function}

There was no rand() function available, so we made one ourselves.

\begin{figure}[!ht]
\begin{code}

static unsigned int next = 1;

int rand_r(unsigned int *seed) {
    *seed = *seed * 1103515245 + 12345;
    return (*seed % ((unsigned int)RAND_MAX + 1));
}

int rand() {
    return (rand_r(&next));
}

\end{code}
\caption{The random function}
\end{figure}


\subsection{Pre-recorded melody}

For the start-up melody, we chose to produce a song using FL Studio, a digital audio workstation.
We would then export the .wav file and convert it to a C array, as a wav file consists of two channels of typically 44,100 samples per second, and 8 bits per sample.
We had to first export the song to a 16-bit int .wav file, as FL Studio didn't accept 8-bit uint .wavformat.
We opened the song in Audacity, as this audio editor allowed us to export it to a 8 bit int .wav file.
Moreover, we used a Python library to convert the .wav samples to a C array, tailoring this general Python .wav interpreter code \cite{wav} to fit our needs. Our edited Python code to produce the C file that holds the two arrays that represents the two channels is shown in figure \ref{fig:python} in appendix A.

The sound clip we made originally lasted around 15 seconds, which results in $\SI{44100}{Hz} * 15 \text{seconds} = 661500 \text{samples}$.
Since this file was too large for the DAC, we shortened the song to 8 seconds.
This wav-file amounted to 704 kB of space, which became 3.5 MB when converted to two arrays in C. This proved to be small enough for the DAC to interpret.

The start up melody can be found here. \cite{song}

\subsection{Button control}

Buttons are set up to control one sound effect each. The remaining buttons will stop the sound currently playing.
When a GPIO interrupt is triggered, we check whether any buttons are being pressed down.
The first button which is currently being pressed will have its designated action executed.
When one sound is initiated, any currently playing sound is cancelled immediately.

\subsection{Energy consumption}

The desire for this task was to be able to use as little power as possible.
To achieve this we put a great deal of work into ensuring the program was put in the most energy saving state possible for the current task.
This means that even though music was played, the card was still in an energy saving mode.

For maximum efficiency the board dives straight into deep sleep after startup.
One could agrue that a small opening sequence would be useful for the end user, but for this task the startup time could instead be instant.
Since there are no definitive answer to this we thought we should go for maximum efficiency this time.

Every interaction with the board gives a feedback.
We made sure that the board goes back to sleep as soon as possible after the given task is done.
When a song is played the interrupt handler wakes up the board and starts to play the song.
If a new song is being played while a current song is playing, the current task is thrown away, and the new song starts.
This technique ensures that no power is wasted on music the end user doesn't want.

When the song is over the card by itself goes back into deep sleep and waits for another interrupt.

The one thing that does use a lot of power is computing generated sounds.
If one wanted to make a more energy efficient card, one could include more songs for playback, and not generate them live.


\newpage

%RESULTS AND TEST
\section{Results and Tests}

\subsection{Sounds}

Three different main sounds were implemented, playing on different button pushes. A button never played the same sound, however, as they were semi-random, given some pre-defined constraints. Our first sound was a coin sound, and given spesific ranges for amplitude, frequency (...), we were able to produce different coin sounds each time the same button is pressed.

\subsection{Comparison}

The comparison of the different values are shown in table \ref{tab:energy_usage}.
One major point one can see from the table is that the biggest difference is between energy mode 1 and energy mode 2. This difference is way lager than energy mode 1 and polling. An interrupt in energy mode 2 is 1617 times less power consuming than polling or interrupts in energy mode 1.
A more detailed view of each energy level is shown in appendix B on page \pageref{appendix:b}.

\begin{table}[ht!]
    \begin{center}
    \begin{tabular}{ | l | l | l | }
        \hline
        I/O Mode    & Idling energy usage \\
        \hline
        Polling & 3.51 mA \\
        \hline
        Interrupts w/Energy mode 1 & 1.19 mA \\
        \hline
        Interrupts w/Energy mode 2 & 2.17 \si{\micro\ampere} \\
        \hline
    \end{tabular}
    \caption{Average current while idling for different methods of input reading}
    \label{tab:energy_usage}
    \end{center}
\end{table}



\newpage

%EVALUATION OF ASSIGNMENT
\section{Evaluation of Assignment}

This assignment was a great introduction to I/O programming in C. In addition to improving our C programming skills, we also learned a lot about sound generation and how the digital/analog conversion is done. The compendium \cite{eeds-compendium} introduced the exercise in an informative way, giving useful step-by-step instructions for how to approach the problem. It also contained useful information about the gcc, hardware access from C code and interrupt handling in C, which we used for solving the problem.

This assignment was a great continuation to the previous assignment, and the C programming felt like a natural way to proceed from the assembly code in the previous assignment.


\newpage

%CONCLUSION
\section{Conclusion}

The LED lights on the gamepad lights up as intended when the corresponding buttons are pushed. We also managed to implement it in an energy efficient way, as required by the assignment. We have, by completing this assignment, gained a greater understanding of not only how to influence the energy efficience, but also of assembly programming, the EFM32 microcontroller, and the GNU toolchain in general. Furthermore, it became clear from the test results that using interrupts and letting the CPU sleep when there was no I/O to be processed was a improvement, in regards to power-consumption, over polling. Especially the deep sleep mode (energy mode 2) was extremely efficient over both regular sleep mode (energy mode 1) and polling. This was also the expected result according to the theory.


\newpage

%REFERENCES
\bibliography{bibtexlibs}{}
\bibliographystyle{plain}
\nocite{*}
All resources checked on 2014-02-09


\newpage

\begin{alphasection}
%APPENDIX A
\section{Appendix A}
\label{appenix:a}

This appendix includes supplimentary tables for the report.

\begin{table}[ht!]
    \begin{center}
    \begin{tabular}{| l | l | l |}
        \hline
        Value   & Mode      & Description \\
        \hline
        0       & STANDARD  & 6 mA drive current \\
        \hline
        1       & LOWEST    & 0.5 mA drive current \\
        \hline
        2       & HIGH      & 20 mA drive current \\
        \hline
        3       & LOW       & 2 mA drive current \\
        \hline
    \end{tabular}
    \caption{EFM32 GPIO drive strength, from page 766 of the reference manual \cite{efm32gg-ref-man}}
    \label{tab:drive_strength}
    \end{center}
\end{table}

\begin{table}[ht!]
    \begin{center}
    \begin{tabular}{ | l | l | l | }
        \hline
        Register    & Alias     & Usage \\
        \hline
        R0          & None      & Variable values\\
        \vdots      & \vdots    & \vdots \\
        R7          & None      & Variable values \\
        \hline
        R8          & cw        & Wait simulation register \\
        \hline
        R9          & gpio\_i   & GPIO\_PC\_BASE \\
        \hline
        R10         & gpio\_o   & GPIO\_PA\_BASE \\
        \hline
        R11         & gpio      & GPIO\_BASE \\
        \hline
    \end{tabular}
    \caption{Register design}
    \label{tab:register_design}
    \end{center}
\end{table}

\begin{table}
    \begin{tabular}{|l|l|l|l|}
    \hline
    ~                    & Polling & Interrupt with  energy mode 1 & Interrupt with  energy mode 2 \\ \hline
    Current while idling & 3.51mA  & 1.13mA                        & 2.17$\mu$                     \\ \hline
    \end{tabular}
    \caption{Comparison}
    \label{tab:comparison}
\end{table}



\newpage

%APPENDIX B
\section{Appendix B}
\label{appenix:b}

This appendix includes supplimentary pictures for the report.

\begin{figure}[ht!]
    \begin{center}
    \includegraphics[width=0.8\textwidth]{assets/img/no_opening_seq.png}
    \caption{eAProfiler reading with no opening sequence. Each spike is a reset.}
    \label{fig:no_opening_seq}
    \end{center}
\end{figure}

\begin{figure}[ht!]
    \begin{center}
    \includegraphics[width=0.8\textwidth]{assets/img/opening_seq.png}
    \caption{eAProfiler reading with opening sequence. The spike is a reset.}
    \label{fig:opening_seq}
    \end{center}
\end{figure}


\end{alphasection}


\end{document}
